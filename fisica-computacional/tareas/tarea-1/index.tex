\documentclass[10pt]{article}
\usepackage{geometry}
\usepackage{multirow}
\usepackage{graphicx}
\usepackage{array}
\usepackage{amssymb}
\usepackage{amsmath}
\usepackage{enumitem}
\usepackage[spanish]{babel}

\title{Tarea 1}
\author{Nicolas Aguilera García - 2127303}
\date{\today}
\geometry{letterpaper, top=2.54cm, bottom=2.54cm, left=3cm, right=3cm}    
\graphicspath{ {./images/} }
\setlength{\parindent}{0cm}
\setlength{\parskip}{0.2em}


\begin{document}
    \maketitle

    \section*{Punto 1}
    \subsection*{a)}
        El copilador \textit{gcc} y \textit{G++} comparten y difieren en los siguientes putos:
        \begin{itemize}
            \item \textbf{Semejanza 1:} \textit{Ambos compiladores son programas que traducen el código fuente de un programa escrito en un lenguaje de programación a un lenguaje de máquina.}
            
            \item \textbf{Semejanza 1:} \textit{Ambos compiladores son capaces de traducir código fuente escrito en C y C++ a código objeto.}
            
            \item \textbf{Diferencia 1:} \textit{El compilador gcc cuenta con más macros predefinidas para compilar código C que las que cuenta por defecto G++.}
            
            \item \textbf{Diferencia 2:} \textit{El compilador gcc admite cualquier formato de archivo de .c y .cpp tratándolos como archivos de C y C++ respectivamente, mientras que G++ aunque puede compilar ambos tipos de archivo, serán tratados como archivos de C++.}
        \end{itemize}

    \subsection*{b)}
        Razones por las que es útil aprender a desarrollar en C:
        \begin{itemize}
            \item Permite realizar un control granular de la memoria, lo que otorga un mejor manejo de los recursos del computador y optimizar bastante el programa.
            \item Se puede compilar para funcionar en la mayoría de plataformas y sistemas operativos.
            \item Es muy utilizado en la industria, lo que hace que cuente con muchas librerías ya desarrolladas que pueden solucionar problemas muy concretos de forma eficiente.
        \end{itemize}

        Razones por las que es útil aprender a desarrollar en Fortran:
        \begin{itemize}
            \item Es un lenguaje con una larga trayectoria y que ha sido usado en muchos proyectos de investigación.
            \item Se encuentra bien documentado y es sencillo iniciar a desarrollar en él.
            \item Cuenta con una gran cantidad de librerías y herramientas que facilitan el desarrollo de programas.
        \end{itemize}

    
    \subsection*{c)}
        La programación es muy útil en Física, ya que nos permite desarrollar herramientas que nos ayuden a entender el mundo que nos rodea. Esto mediante simulaciones de sistemas complejos o mediante el desarrollo de algoritmos que nos permitan resolver problemas de manera rápida y eficiente.
    
        De la materia, espero que esta me permita adquirir conocimientos sobre como poder aplicar los conocimientos de programación en la solución de problemas físicos, y que el aprendizaje se realice mediante ejemplos prácticos y no tan solo mediante teoría.





    \section*{Punto 2}
    The total Mie-scattering cross section per particle is given by,

    \begin{equation}
        \alpha_{ext} \left( \phi, m \right) = \pi r^2Q_{ext}\left( \phi, m \right)
    \end{equation}

    where $ Q_{ext}(\phi, m) $ is the extinction efficiency factor of a particle with radius $ r $, which depends on the refractive index $ m $ and the dimensionless size parameter $ \phi = \frac{2 \pi r}{\lambda} $, being $ \lambda $ the incident wavelength.

    Additionally, based on the Mie theory, it is possible to estimate the total extinction with the contribution of particles in a define range in the atmospheric column through the Aerosol Optical Depth (AOD), employing the following expression,

    \begin{equation}
        AOD\left(\lambda,\phi,r \right) = \int_{0}^{r_{ref}} \pi r^2 Q_{ext}\left(\phi, m \right) \, n\left(r \right) \, dr
    \end{equation}

    where $ n(r) $ denotes the sum of n log-normal size distribution of particles as follow \cite{cite},

    \begin{equation}
        n\left(r \right) = \sum_{i = 1}^{n} \frac{N_{i}}{(2\pi)^{1/2} \log\sigma_{i}} \exp \left[-\frac{(\log r - \log \bar{r}_{i})^2}{2 \log^2\sigma_{i}}  \right]
        \label{eq:eq1}
    \end{equation}

    where is $ N_{i} $ the particle concentration number with radius $ r $, $ \sigma_{i} $ is the standard deviation of the $ i^{th} $ log normal mode, and $ \bar{r}_{i} $ is the mean diameter.


    \section*{Punto 3}

    A continuación se presenta la tabla \ref{tab:tab1} con los símbolos del abecedario griego junto con algunas notaciones matemáticas útiles. Además, se referencia la ecuación \ref{eq:eq1} requerido y escrita con anterioridad en el texto.

    \begin{table}[h]
    \centering
    \begin{tabular}{|m{0.6cm}|m{0.6cm}|m{0.6cm}|m{0.6cm}|c|} 
        \hline
        \multicolumn{4}{|c|}{Abecedario Griego} & \multirow{2}{*}{Símbolos Matemáticos}  \\ 
        \cline{1-4}
        \multicolumn{2}{|c|}{Minúsculas} & \multicolumn{2}{c|}{Mayúsculas} & \\ 
        \hline
        $\alpha$ & $\nu$      & $A$      & $M$      & $\oint_{a}^{b}$     \\ 
        \hline
        $\beta$      & $\xi$      & $B$      & $N$      & $\frac{\partial^2 }{\partial x^2}$     \\ 
        \hline
        $\gamma$      & $o$      & $\Gamma$      & $\Xi$      & $\overline{AB}$     \\ 
        \hline
        $\delta$      & $\pi$      & $\Delta$      & $O$      & $A \rightarrow B$     \\ 
        \hline
        $\epsilon$      & $\rho$      & $E$      & $\Pi$      & $\pi \approx 3.14$     \\ 
        \hline
        $\zeta$      & $\sigma$      & $Z$      & $P$      & $\iint_{a}^{b}$     \\ 
        \hline
        $\eta$      & $\tau$      & $H$      & $\Sigma$      & $x \propto y$     \\ 
        \hline
        $\theta$      & $\upsilon$      & $\Theta$      & $T$      & $x \in \mathbb{R}$     \\ 
        \hline
        $\iota$      & $\phi$      & $I$      & $\Upsilon$      & $y \pm \bigtriangleup y$     \\ 
        \hline
        $\kappa$      & $\chi$      & $K$      & $\Phi$      & $\lim_{x\rightarrow a}$     \\ 
        \hline
        $\lambda$      & $\psi$      & $\Lambda$      & $X$      & $$     \\ 
        \hline
        $\mu$ & $\omega$ & $\Psi$ & $\Omega$ & $$ \\
        \hline
    \end{tabular}
    \caption{Uso de sintaxis matemática en \LaTeX
    \label{tab:tab1}}
    \end{table}



    \begin{thebibliography}{9}
        \bibitem{cite} 
        P. V. Hobbs, Aerosols-cloud-climate interactions, Vol, 54, Academic Press, 1993.
    \end{thebibliography}

    
\end{document}
